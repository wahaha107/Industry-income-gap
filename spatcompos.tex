\documentclass[12pt, titlepage]{article}

% \linespread{1}
\usepackage{hyperref}
\usepackage{mathrsfs}
\usepackage{url}
\usepackage{xcolor}
\usepackage{natbib}
\usepackage[margin=1in]{geometry} 
\usepackage{amsmath,amsthm,amssymb,bm,amsmath}
\usepackage{graphicx}
\usepackage{mathtools}
\usepackage{setspace}

\newcommand{\N}{\mathbb{N}}
\newcommand{\Z}{\mathbb{Z}}
\newtheorem{proposition}{Proposition}


\usepackage{color}
\newcommand{\blue}[1]{{\textcolor{blue}{#1}}}
\newcommand{\red}[1]{{\textcolor{red}{#1}}}


\title{Bayesian Hierarchical Spatial Regression Model for Compositional Data with Applications to Intrasectoral Earnings Inequality in China}
\author{
  Guanyu Hu
  \footnote{Department of Statistics, University of Connecticut},
  \and
  Tao Wang
  \footnote{School of Statistics, Shanxi University of Finance and
    Economics, China},
  \and
  Jun Yan
  \footnote{Department of Statistics, University of
    Connecticut}
}
                
% \date{}


\begin{document}
\maketitle


\begin{abstract}
Intrasectoral earnings inequality is of great economic, societal, and
political importance. The sectoral contributions to the inequality as
measured by the Kuznets ratio or the maximum equalization percentage
are represented by the proportion of their contribution. Regression
analyses of such data are complicated by the spatial structure over
provinces. We propose a hierarchical regression model for the
spatially dependent compositional data of 19 sectors over 30 provinces
in China. The properties of the proposed model are examined and
inferences are performed using Markov chain Monte Carlo. Two Bayesian
model comparison criteria, the Deviance Information Criterion and the
Logarithm of the Pseudo-Marginal Likelihood, are used for model
comparison. Extensive simulation studies are carried out to assess the
empirical performance of the proposed methods. Application to the 2017
provincial data reveals interesting findings.

\bigskip
\noindent\textbf{Keywords}:
Conditional Autoregressive;
Intrasectoral Earnings Inequality;
Spatial Dirichlet Regression 
\end{abstract}

\doublespacing 


\section{Introduction}


\red{JY: no hard spacing control or newpage in general}

\red{JY: read Jieying's writing tips, please.}

\red{JY: introduction is a place where a lot of references should be
  reviewed. Usually there is no need to get to notations.}

The traditional income gap indicator can only measure the industry
income gap comprehensively, but can not understand the internal cause
of the income gap, that is, do not appreciate the constribution of
each industry to the overall income inequality. By the derivation and
transformation of the Kuznets index \citep{kuznets1955economic}, a
decomposable industry
income gap whose the constribution rate meets the requirements of the
compositional data is constructed in this paper to systematically
understand the contribution rates of income gaps in various
industries.
\red{JY: needs references to a review of income inequality measures.}


\red{JY: add references in bib file}
The concept of compositional data originated from the work of Ferrers in the 19th century. It refers to any non-negative vector $X=(X_{1},X_{2},\ldots,X_{D})$ , the $D$ components of $X$ satisfy the following conditions: $$\sum_{i=1}^DX_{i}=1,0\leq X_{i}\leq 1.$$
Due to its important characteristics—unit-sum constraint and
non-negative constraints, compositional data is often used as a tool
for analyzing problems such as proportions and structures, and is
widely used in social, economic, and technological fields. At the same
time, due to the unit-sum constraint, a certain data changes are
required to eliminate the redundant degree of freedom when analyzing
the actual problems by using the compositional data.


Three kinds of compositional data change methods are adopted for
analysis and prediction in this paper. First, two prediction methods,
namely asymmetric logratio prediction and grey prediction, are used to
predict. Then, combination forecasting method is used to combine two
single prediction methods to make more accurate prediction of internal
structural changes in the industry income gap.

\section{Data}
\subsection{Sources of Data}
The data used in this paper are all from China Statistical Yearbook 2017 published on the website of the National Bureau of Statistics of China. The data contains 3 indicators——the number of employees in urban units by industry (10,000 persons), the total wages of employees in urban areas by industry (100 million yuan) and the average salary (yuan)——and covers 19 industries——according to the official standards of the National Bureau of Statistics, China's national economic industry classification is divided into 19 categories, which are subordinate to the primary, secondary and tertiary industries. 
The primary industry is agriculture, forestry, animal husbandry and fishery. There are four industries in the secondary industry, namely mining, manufacturing,  construction, production and supply of electricity, heat, gas and water, while the tertiary industry is the remaining 14 industries.

\subsection{Indicator Selection}

There are lots of reasons for the industry income gap, under the Paradigm of neoclassical economic theory, factors are classified into two categories. The first is the difference in individual characteristics of laborers, such as health status and education level (Juhn 1994), and the second is the difference in industry characteristics, such as labor intensity and risk level, also known as "compensatory wage difference" (Dickens and Kate, 1971).

However, the reality of the complete competition hypothesis described by neoclassical economic theory does not exist.in the actual research question, the influencing factors of industry income gap are various, such as industry difference, industry monopoly, education level, social relations, and some macro-economic indexes can also cause the income gap between different regional industry, therefore, according to the relevant research, this paper sums up the indexes that affect the income gap of industries in different regions. Specific indicators and explanations are as follows:

Per capita GDP: The total GDP of each region in 2016 divided by the total population;

Per capita fixed asset investment: Total fixed assets by region divided by total population;

FDI proportion: FDI divided by GDP of each region;

Foreign Trade Dependency: Total imports and exports of each region divided by GDP;

Level of transport facilities: total freight transport in each region divided by total population;

Ratio of tertiary industry: the added value of tertiary industry in each region divided by GDP;

Local economic intervention: regional fiscal expenditure divided by GDP;

Average education attainment: average of all population years of education;

Proportion of people with bachelor's degree or above: the total number of people with bachelor's degree divided by total population;

Urbanization rate: urban population divided by total population.


\subsection{Indicator Decomposition}

The Kuznets index is derived from Kuznets' interpretation of economic
development and income disparity. The horizontal coordinate of the
Kuznets curve is the growth of wealth per capita, the ordinate is the
distribution of wealth per capita, and the Kuznets curve presents an
inverted U shape in the coordinate system. The kuznets ratio is a
method to measure the relationship between economic development and
income distribution, it is calculated by the following
formula:$R=\sum_{i=1}^n\left|y_{i}-p_{i}\right|$, where $R$ is the
Kuznets ratio, $y_{i}$ and $p_{i}$ respectively represent the
proportion of income and population of each comparison group. The
larger the value is, the more unequal the income is, and vice
versa. However, this form cannot understand the relationship of the
internal proportional structure, so we deduct and transform the
Kuznets index.

Suppose
that $$R=\sum_{i=1}^n\left|\frac{x_{i}}{X} - \frac{k_{i}}{K}\right|$$
where $n$ represents the number of industries, $x_{i}$ represents the
total income of the i-th industry, $X$ is the sum of the income of all
industries, $k_{i}$ is the sum of the employment of the u-th industry,
and $K$ is the total number of the employment in the whole
industries. $y_{i}=\frac{x_{i}}{X}$ is the proportion of the income of
the i-th industry to the total income, and $p_{i}=\frac{k_{i}}{K}$ is
the i-th industry employment as a proportion of the total employment
population. The further deduction process is
$$R=\sum_{i=1}^n\left|\frac{x_{i}}{X}-\frac{k_{i}}{K}\right|=\sum_{i=1}^n\frac{k_{i}}{K}\left|\frac{x_{i}}{X}\times\frac{k}{k_{i}}-1\right|$$
$$=\sum_{i=1}^n\frac{k_{i}}{K}\left|\frac{x_{i}}{k_{i}}/\frac{X}{K}-1\right|=\frac{K}{X}\sum_{i=1}^n\frac{k_{i}}{K}\left|\frac{x_{i}}{k_{i}}-\frac{X}{K}\right|$$
$$=\frac{1}{\overline{g}}\sum_{i=1}^np_{i}\left|\overline{g_{i}}-\overline{g}\right|$$
Among them, $\overline{g}=\frac{X}{K}$ is the average salary for all
industries, $\overline{g_{i}}=\frac{x_{i}}{k_{i}}$ represents the
average salary in the i-th industry. $R$ is made of two parts after
the deduction. One part is $\frac{1}{\overline{g}}$, it is the average
wage level of all industries. Its role in the formula is mainly
dimensionless, and the impact on the inequality index is not
obvious. The other part is
$\sum_{i=1}^np_{i}\left|\overline{g_{i}}-\overline{g}\right|$, it is
an important reason for influencing the inequality
index. $\left|\overline{g_{i}}-\overline{g}\right|$ is the absolute
difference between the average wage of the i-th industry and the total
average wage. $p_{i}=\frac{k_{i}}{K}$ is the proportion of employment
in the i-th industry, that is, the weight of the i-th industry
inequality.



The income gap contribution rate of each industry in the whole can be
obtained by decomposing the Kuznets index. If $m_{i}$ means the income
gap contribution rate of the i-th industry, it can be calculated by
\[
  m_{i}=p_{i}\left|\overline{g_{i}}-\overline{g}\right|/\sum_{i=1}^np_{i}\left|\overline{g_{i}}-\overline{g}\right|,0\leq
  m_{i}\leq 1,\sum_{i}^nm_{i}=1.
\]
The contribution rate $m_{i}$ is the measure indicator calculated
after taking the absolute value of each group. The absolute
contribution rate satisfies the basic requirements of the
compositional data—non-negative and $\sum_{i}^nm_{i}=1$. Then, the
structure of China's industry income gap is analyzed and predicted by
using the prediction method of the compositional data.

\section{Methodology}
\subsection{Dirichlet Regression for Composition Data}
\subsection{Conditional Autoregressive Model}
\subsection{Hierarchal Spatial Dirichlet Regression Model}

\section{Bayesian Computation}
\subsection{Prior Specification and Posterior Distribution}
\subsection{The MCMC Sampling Schemes}
\subsection{Bayesian Model Assessment}
\section{Simulation}
\section{Real Data Analysis}
\section{Discussion}

\bibliographystyle{asa}
\bibliography{spatcompos} 

\end{document}
